\documentclass[10pt]{beamer}

\usetheme[progressbar=frametitle]{metropolis}
\usepackage{appendixnumberbeamer}

\usepackage{booktabs}

% For removing Figure 1 in figures
\usepackage{caption}

%For linking email and website
\usepackage{hyperref}

%For enumeration as words
\usepackage{blindtext}
\usepackage{enumitem}
\usepackage{geometry}

\usepackage{tikz}
\usepackage{xspace}


\title{Theory of Computation}
\subtitle{Tutorial - Languages}
\author{Cesare Spinoso-Di Piano}
\date{}

\begin{document}

\maketitle

\begin{frame}{Plan for today}
    \setbeamertemplate{section in toc}[sections numbered]
    \tableofcontents[hideallsubsections]
\end{frame}

\section{Languages}

\begin{frame}{Definitions}
    \textbf{Definition.} An \textbf{alphabet} $\Sigma$ is a finite set of symbols. It must contain at least one symbol.
    \begin{itemize}
        \item \textbf{Example 1.} $\Sigma = \{a,b,...,z\}$
    \end{itemize}
    \textbf{Definition.} A \textbf{string} is a sequence of symbols from a given alphabet $\Sigma$.
    \begin{itemize}
        \item \textbf{Example 2.} Given $\Sigma = \{a,b\}$, $w = abba$ is a well-defined string over $\Sigma$, but $v = acb$ is not. %c is not from this alphabet
    \end{itemize}
    \textbf{Definition.} A \textbf{language} $L$ is a set of strings defined over an alphabet $\Sigma$.
    \begin{itemize}[itemsep=5mm, parsep=0pt]
        \item \textbf{Example 3.} Given $\Sigma=\{a,b\}$, $L=\{a,aa,aab\}$ is a finite language.
        \item \textbf{Example 4.} Given $\Sigma=\{a,b\}$, $L=\{a^n : n > 1\}$ is an infinite language.
    \end{itemize}
\end{frame}
\begin{frame}{Operations on strings}
    Given strings $w = a_1a_2\dots a_n$ and $v = b_1b_2 \dots b_m$ defined over $\Sigma$, define the following string operations
    \begin{itemize}[itemsep=5mm, parsep=0pt]
        \item \textbf{String concatenation: } $ wv = a_1a_2\dots a_n b_1b_2 \dots b_m$
        \item \textbf{String reversal: } $w^R = a_n \dots a_2a_1$
        \item \textbf{String length: }The number of characters in a string. $|w| = n$ and $|v| = m$

    \end{itemize}
\end{frame}


\begin{frame}{Empty string}
    \textbf{Definition.} The empty string $\lambda$ (also $\varepsilon$) is defined as the string with length 0. $|\lambda| = 0$. Equivalent to `' in Java/Python.
    \begin{itemize}[itemsep=5mm, parsep=0pt]
        \item \textbf{Example 1.} What is $\lambda^R = $?
        \item \textbf{Example 2.} What is $w \lambda$ for any string $w$?

    \end{itemize}
\end{frame}

\begin{frame}{More operations on strings}
    \textbf{Definition.} For every string $v = b_1b_2 \dots b_m$ defined over $\Sigma$, $v^n = \underbrace{ v \cdot v \cdot ...\cdot v}_{\text{n times}}$.
    \begin{itemize}[itemsep=5mm, parsep=0pt]
        \item \textbf{Example 1.} What is $v^0 = $?
        \item \textbf{Example 2.} $w = ab, v = ba$. $w^2v^2w^0 = ?$
    \end{itemize}
\end{frame}

\begin{frame}{Substring}
    \textbf{Definition.} A string $z$ is a \textbf{substring} of a string $w$ if it appears consecutively within $w$.
    \begin{itemize}
        \item \textbf{Example 1.} Let $w = abaa$, what are its possible substrings?
        \item \textbf{Example 2.} Given a string $w$ of length $n$, how many substrings will it have?
    \end{itemize}
\end{frame}

\begin{frame}{Prefix and suffix}
    \textbf{Definition.} A string $x$ is a \textbf{prefix} of $w$ if there exists a string $z$ such that $w = xz$.

    \textbf{Definition.} A string $x$ is a \textbf{suffix} of $w$ if there exists a string $z$ such that $w = zx$.

    \begin{itemize}
        \item \textbf{Example 1.} Let $w = abbaa$, what are the prefixes and suffixes of $w$?
        \item \textbf{Example 2.} True or False. For any string $w$ there is exactly one substring $x$ that is both a prefix and a suffix.
        \item \textbf{Example 3.} Let $w$ be a string of length $n$, how many prefixes will $w$ have?
    \end{itemize}

\end{frame}


\begin{frame}{Operations on languages}
    \textbf{Definition.} The \textbf{union}, \textbf{intersection} and \textbf{difference} of languages can be applied as set operations.\\

    \begin{itemize}
        \item \textbf{Example 1.}         Given $\Sigma = \{a,b\}$, $L_1 = \{a, ab, abab\}$, $L_2 = \{(ab)^n: n \geq 0\}$, what is:
            \begin{itemize}[itemsep=5mm, parsep=0pt]
                \item $L_1 \cup L_2 = $
                \item $L_1 \cap L_2 = $
                \item $L_1 - L_2 = $
                \item $L_2 - L_1 = $
            \end{itemize}
            \vspace{5mm}
    \end{itemize}

\end{frame}

\begin{frame}{Operations on languages - Continued}
    \textbf{Definition.} The \textbf{reverse} of a language is defined as $L^R = \{w^R : w \in L\}$.
    \begin{itemize}[itemsep=5mm, parsep=0pt]
        \item \textbf{Example 2.} Given $L_1 = \{a, ab, abab\}$, $L_2 = \{(ab)^n: n \geq 0\}$.
        \item What is $L_1^R = $
        \item What is $L_2^R = $
    \end{itemize}
\end{frame}

\begin{frame}[t]{Operations on languages - Continued}
    \textbf{Definition.} Given $L_1$, $L_2$ the language \textbf{concatenation} $L_1L_2$ is defined as $\{wv : w \in L_1, v \in L_2\}$ (like the cross-product of two sets).
    \begin{itemize}
        \item \textbf{Example 1.} Given $L_1 = \{a\}$, $L_2 = \{a^n: n \geq 0\}$, what is:
            \begin{itemize}[itemsep=10mm, parsep=0pt]
                \item $L_1L_2 = $
                \item $L_2L_1$ =
            \end{itemize}
        \item \textbf{Example 2.} In general, for two languages $L_1, L_2$, is $L_1L_2 = L_2L_1$?
    \end{itemize}
\end{frame}

\begin{frame}[t]{Operations on languages - Continued}
    \textbf{Definition.} Given a language $L$, define $L^n = \underbrace{LL\dots L}_{\text{n times}}$.
    \begin{itemize}[itemsep=10mm, parsep=0pt]
        \item \textbf{Example 1.} Let $L_1 = \{a^n b^n : n \geq 0\}$. What is $L_1^2 $?
        \item \textbf{Example 2.} What is $L^0 = $?  Give some intuition behind this.
    \end{itemize}
\end{frame}

\begin{frame}[t]{Operations on languages - Continued}
    \textbf{Definition.} Given a language $L$, the \textbf{star-closure} of $L$, denoted by $L^*$, is defined as the following language: $L^* = L^0 \cup L \cup L^2 \cup L^3 \cup \dots$
    \begin{itemize}[itemsep=15mm, parsep=0pt]
        \item \textbf{Example 1.} Given $\Sigma = \{a,b\}$, what is the set of all possible strings over $\Sigma$ ? .
        \item \textbf{Example 2.} Given $L_4 = \{a,ab\}$ how many strings of length 0, 1 and 2 are there in $L_4^*$? What are those strings?

    \end{itemize}
\end{frame}

\begin{frame}[t]{Operations on languages - Continued}

    \textbf{Defintion.} Given a language $L$, the \textbf{positive-closure} is defined as $L^+ = L^1 \cup L^2 \cup L^3 \cup \dots$
    \begin{itemize}[itemsep=10mm, parsep=0pt]
        \item \textbf{Example 1.}  Given $L_4 = \{a,ab\}$, what is the shortest string in $L_4^+$?\\
        \item \textbf{Example 2.} True or False. $L^+ = L^* - \{\lambda\}$?\\

    \end{itemize}
\end{frame}

\begin{frame}[t]{Three very important languages}
    Three languages that we will OFTEN see/use in the proofs/counter-examples: $\emptyset$, $\{\lambda\}$, $\Sigma^*$.
    \begin{itemize}[itemsep=5mm, parsep=0pt]
        \item \textbf{Example 1.} Given $L = \{a, ab\}$ over $\Sigma = \{a,b\}$. How do we define $\overline{L}$?\\
        \item \textbf{Example 2.} What is $L\emptyset$?\\
        \item \textbf{Example 3.} What is $\emptyset^0$?\\
        \item \textbf{Example 4.} What is $\emptyset^*$?\\

    \end{itemize}
\end{frame}

\begin{frame}[t]{Exercises}
    Let $L_1 = \{a^nb^{n+1} : n \geq 0\}$, $L_2 = \{a^n : n \mod 2 = 0\}$ then:
    \begin{itemize}[itemsep=5mm, parsep=0pt]
        \item \textbf{a.} $L_1 - L_2 = $?\\
        \item \textbf{b.} $\Sigma^* \cup L_1^R\overline{L_2}\{\lambda\}=$?\\
        \item \textbf{c.} What is the shortest string in $L_1^0\{a,b\}\emptyset$?\\
    \end{itemize}
\end{frame}

\begin{frame}[t]{Exercises - Continued}
    \begin{itemize}[itemsep=30mm, parsep=0pt]
        \item \textbf{d.} Find languages $L_1, L_2$ such that $L_1L_2 = L_1$?\\

        \item \textbf{e.} Find a language $L$ such that $L^* = L$?\\
    \end{itemize}
\end{frame}

\begin{frame}[t]{Exercises - Continued}
    \begin{itemize}[itemsep=15mm, parsep=0pt]
        \item \textbf{f.} \textbf{True or False.} $|L \cup L^2| = |L| + |L^2|$.
        \item \textbf{g.} \textbf{True or False.} $|L \cup L^2| \leq |L| + |L^2|$.
    \end{itemize}
\end{frame}

\end{document}