\documentclass[10pt]{beamer}

\usetheme[progressbar=frametitle]{metropolis}
\usepackage{appendixnumberbeamer}

\usepackage{booktabs}

% For removing Figure 1 in figures
\usepackage{caption}

% For enumeration as words
\usepackage{blindtext}
\usepackage{enumerate}
\usepackage{geometry}

\usepackage{tikz}
\usepackage{xspace}


\title{Theory of Computation}
\subtitle{Tutorial - The Pumping Lemma (for Regular Languages)}
\author{Cesare Spinoso-Di Piano}
\date{}

\begin{document}

\maketitle

\begin{frame}{Plan for today}
    \setbeamertemplate{section in toc}[sections numbered]
    \tableofcontents[hideallsubsections]
\end{frame}


\section{Non-regular languages}

\begin{frame}{What is a ``non-regular'' language?}

\end{frame}

\section{The Pumping Lemma}

\begin{frame}{The Pumping Lemma}

    \textbf{Lemma.} Let $L$ be an infinite regular language. Then there exists some positive integer $m$ such that any $w \in L$ with $|w| \geq m$ can be decomposed a $w = xyz$ with $|xy| \leq m$, and
    $|y| \geq 1$ such that $w_i = x{y^i}z$ is also in $L$ for all $i = 0, 1, 2, \dots$.

\end{frame}

\begin{frame}{Breaking Down the PL}
    % \begin{align*}
    %      & L \text{ regular }                       \\
    %      & \implies                                 \\
    %      & \exists m > 0                            \\
    %      & \forall w \in L, |w| \geq m              \\
    %      & \exists w = xyz, |xy| \leq m, |y| \geq 1 \\
    %      & \forall i \geq 0, xy^iz \in L
    % \end{align*}
\end{frame}

\section{Using the PL}

\begin{frame}{Using the PL}
    \textbf{Idea.} If $L$ regular then $L$ satisfies the conditions of the PL $\equiv$ If $L$ does not satisfy the conditions of the PL then $L$ is \textit{not} regular.

    What is the negation of the conditions of the PL ?
\end{frame}

\begin{frame}{Using the PL}
    Imagine a game where \textit{you} play against some \textit{opponent}. The point of the game is to win, i.e. to fool your opponent. The game follows the following 4 moves.

    \begin{enumerate}[Move 1.]
        \item \textbf{Opponent's move.} Chooses an $m > 0$. You have no knowledge of what this $m$ could be.
        \item \textbf{Your move.} Pick a \textit{particular} string $w$ with $|w| \geq m$.
        \item \textbf{Opponent's move.} They decompose your string into $xyz$ where $|xy| \leq m$, $|y| > 0$. There are many possible decompositions, you don't know which one your opponent picked.
        \item \textbf{Your move.} You pick a \textit{particular} $i \geq 0$ and show that $xy^iz \notin L$. You have defeated your opponent!
    \end{enumerate}
\end{frame}

\begin{frame}[t]{Exercise}
    \textbf{Exercise.} Prove that $L=\{{0^n}1^j, n>j \geq 0\}$ is not regular.
\end{frame}

\begin{frame}[t]{Exercise}
    \textbf{Exercise.} Prove that $L = \{a^n : n = k^3\}$ is not regular.
\end{frame}

\begin{frame}[t]{Exercise}
    \textbf{Exercise.} Is the language $L=\{a^nb^l:n/l\in Z\}$ regular? Prove your claim.
\end{frame}

\begin{frame}[t]{Exercise}
    \textbf{Exercise.} Is the language $L=\{w \in \{a,b\}^* : n_a(w) + n_b(w) = 4n_b(w) \}$ regular? Prove your claim.
\end{frame}

\begin{frame}[t]{Exercise}
    \textbf{Exercise.} Is the language $L = \{a^{2n} : n \geq 0\} \cup \{a^{2^n} : n \geq 0 \}$ regular? Prove your claim.
\end{frame}

\begin{frame}[t]{Exercise}
    \textbf{Exercise.} Let $L = \{a^nb^kc^{n+k}d^p : n,k,p \geq 0\}$ be a language we are trying to show is not regular using the pumping lemma. Suppose your opponent chooses an integer $m > 0$. Which of the following strings for $w$ would be a suitable choice to show that $L$ is not regular? Should you pump up or pump down?
    \begin{itemize}
        \item[a.] $w = a^mb^mc^m$
        \item[b.] $w = a^mb^m$
        \item[c.] $w = a^mc^m$
        \item[d.] $w = a^mb^{2m}c^md^m$
    \end{itemize}
\end{frame}

\begin{frame}[t]{Exercise}
    \textbf{Exercise.} Show that you cannot use the pumping lemma to prove that $L = \{a^ib^jc^k : i = 0 \implies j = k\}$ is not regular.
\end{frame}

\section{Using the closure properties (again)}

\begin{frame}{Using the closure properties}
    \textbf{Idea.} If you are trying to show that $L$ is not regular, assume (for contradiction) that it is regular. Now ``manipulate'' $L$ using the closure properties to conclude that another language $L'$ is regular \textit{when you know that this is not the case}.

    \textbf{Example.} Consider the language $L = \{ w \in \{a, b\}^* : n_a(w) = n_b(w) \}$. We claim this language is not regular. Suppose (for contradiction) that it is regular. Then, $L' = L \cap L(a^*b^*)$ must be regular by the closure properties. But this would mean that $L' = \{a^nb^n : n \geq 0\}$ is regular, a contradiction.
\end{frame}

\begin{frame}[t]{Exercise}
    \textbf{Exercise.} Prove that $L=\{w\in \Sigma ^*: w$ has more $0's$ than $1's \}$ is not regular.
\end{frame}

\begin{frame}[t]{Exercise}
    \textbf{Exercise.} Use the closure properties of regular languages to prove that $L=\{{a^n}b^n: n\geq 335\}$ is not regular.
\end{frame}

\begin{frame}[t]{Exercise}
    \textbf{Exercise.} Is the language $L = \{w_1cw_2 : w_1,w_2 \in \{a,b\}^*, w_1 \neq w_2\}$ regular? Prove your answer.
\end{frame}

\end{document}
