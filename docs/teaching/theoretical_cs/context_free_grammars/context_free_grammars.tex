\documentclass[10pt]{beamer}

\usetheme[progressbar=frametitle]{metropolis}
\usepackage{appendixnumberbeamer}

\usepackage{booktabs}

% For removing Figure 1 in figures
\usepackage{caption}

% For enumeration as words
\usepackage{blindtext}
\usepackage{enumerate}
\usepackage{geometry}

\usepackage{tikz}
\usepackage{xspace}


\title{Theory of Computation}
\subtitle{Tutorial - Context-Free Grammars and Languages}
\author{Cesare Spinoso-Di Piano}
\date{}

\begin{document}

\maketitle

\begin{frame}{Plan for today}
    \setbeamertemplate{section in toc}[sections numbered]
    \tableofcontents[hideallsubsections]
\end{frame}

\section{Context-Free Grammars and Languages}

\begin{frame}{Context-Free Grammars}
    \textbf{Definition.} A \textbf{context-free} grammar $G = (V,T,S,P)$ is a grammar where all the production rules have the form
    \begin{align*}
        A & \rightarrow x
    \end{align*}
    Where $A \in V$ and $x \in (T \cup V)^*$.\\
    \textbf{Example.} Which of the following production rules could be seen in a CFG $G = (\{S,A\},\{a,b\},S,P)$?
    \begin{itemize}
        \item $S \rightarrow aaA | abS$
        \item $S \rightarrow aAa | aASb$
        \item $aSa \rightarrow SA$
    \end{itemize}
\end{frame}

\begin{frame}{Context-Free Languages}
    \textbf{Definition.} A language $L$ is \textbf{context-free} if and only if there is a context-free grammar $G$ such that $L(G) = L$.
    \begin{itemize}
        \item Are regular languages also context-free languages?
        \item Are context-free languages also regular?
    \end{itemize}
\end{frame}

\begin{frame}{Example}
    \textbf{Example.} What language is generated by the following CFG $G = (\{S\}, \{a,b\}, S, P)$ where $P$ is
    \begin{align*}
        S & \rightarrow aSa | bSb | \lambda
    \end{align*}
    List the generated strings from shortest to longest:
    \begin{itemize}
        \item $S \Rightarrow \lambda$
        \item $S \Rightarrow aSa \Rightarrow aa$
        \item $S \Rightarrow bSb \Rightarrow bb$
        \item $\dots$
        \item Any string of the form $w = \textcolor{red}{a_1} \textcolor{blue}{a_2} \textcolor{green}{a_3} \dots  \textcolor{green}{a_3} \textcolor{blue}{a_2} \textcolor{red}{a_1}$. That is $w$ is any palindrome of (even or odd ?) length.
    \end{itemize}
\end{frame}

\begin{frame}[t]{Exercise}
    \textbf{Exercise.} Which of the following string(s) can be generated by the following grammar $G = (\{S,A,B\}, \{a,b\}, S, P)$, where P is
    \begin{align*}
        S & \rightarrow AaaB | aaB \\
        A & \rightarrow AB         \\
        B & \rightarrow aba | S
    \end{align*}
    \begin{enumerate}[a.]
        \item aabb
        \item aaba
        \item aaaba
        \item baba
    \end{enumerate}
\end{frame}

\begin{frame}[t]{Exercise}
    \textbf{Exercise.} Give a grammar $G$ that generates the language $L = \{a^nb^kc^l : l = 2k + n, k,l,n \geq 0\}$. Is this language context-free?
\end{frame}

\begin{frame}[t]{Exercise}
    \textbf{Exercise.} Give a context-free grammar $G$ that generates the language $L = \{a^nb^m : m - n = 1\}$.
\end{frame}

\begin{frame}[t]{Exercise}
    \textbf{Exercise.} Give a context-free grammar $G$ that generates the language $L = \{uv^Rwvu : u,v,w \in \{a,b\}^*, |u| = 2, |w| = 3 \}$.
\end{frame}

\begin{frame}[t]{Exercise}
    \textbf{Exercise.} Give a context-free grammar $G$ that generates the language $L = \{uawb : u,w \in \{a,b\}^*, |u| = |w| \}$.
\end{frame}

\section{Derivation Order}

\begin{frame}{Derivation Order}
    \textbf{Definition.} A derivation sequence is said to be leftmost if in each step the leftmost variable in the sentential form is replaced. If in each step the rightmost variable is replaced, we call the derivation sequence rightmost.
\end{frame}

\begin{frame}{Example}
    \textbf{Example.} $G = (\{S,A,B\}, \{a,b\}, S, P)$ where $P$ is
    \begin{align*}
        S & \rightarrow AB          \\
        A & \rightarrow aaA|\lambda \\
        B & \rightarrow Bb|\lambda
    \end{align*}

    The following is a leftmost derivation for $aab$ $S \Rightarrow \textcolor{red}{A}B \Rightarrow aa\textcolor{red}{A}B \Rightarrow aa\textcolor{red}{B} \Rightarrow aa\textcolor{red}{B}b \Rightarrow aab$

    The following is a rightmost derivation for $aab$ $S \Rightarrow A\textcolor{red}{B} \Rightarrow A\textcolor{red}{B}b \Rightarrow \textcolor{red}{A}b \Rightarrow aa\textcolor{red}{A}b \Rightarrow aab$
\end{frame}

\begin{frame}[t]{Exercise}
    \textbf{Exercise.} The following grammar generates the language $L(0^*1(0+1)^*)$. Give leftmost and rightmost derivation sequences for the string 00101.

    $G = (\{S,A,B\}, \{0,1\}, S, P)$ where $P$ is
    \begin{align*}
        S & \rightarrow A1B           \\
        A & \rightarrow 0A|\lambda    \\
        B & \rightarrow 0B|1B|\lambda
    \end{align*}

\end{frame}

\section{Ambiguity}

\begin{frame}{Ambiguity}

    \textbf{Definition.} A grammar $G$ is \textbf{ambiguous} if for a string $w \in L(G)$ there exists two \textbf{\underline{distinct}} leftmost (or rightmost) derivation sequences of $w$.

    A visual alternative to using derivation order is using \textbf{grammar parse trees}.
\end{frame}

\begin{frame}{Example}
    \textbf{Example.} The grammar $G = (\{S\}, \{b\}, S, P)$ where $P$ is
    \begin{align*}
        S & \rightarrow bS | bb | \lambda
    \end{align*}
    is \textbf{ambiguous}. Since for $bb \in L(G)$ we have two leftmost (can also consider them rightmost) derivation sequences
    \begin{align*}
        S & \Rightarrow bb                                \\
        S & \Rightarrow bS \Rightarrow bbS \Rightarrow bb
    \end{align*}
\end{frame}

\begin{frame}[t]{Exercise}
    \textbf{Exercise.} Show that the following grammar is ambiguous: $G = (\{S, A, B\},  \{a,b\}, S, P)$ where $P$ is
    \begin{align*}
        S & \rightarrow AB|aaaB \\
        A & \rightarrow a|Aa    \\
        B & \rightarrow b
    \end{align*}

\end{frame}

\end{document}